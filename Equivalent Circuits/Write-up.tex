\documentclass[12pt, a4paper]{article}
\usepackage{import}
\usepackage{preamble}

\title{Equivalent Circuits}
\date{\(10^\mathrm{{th}}\) November 2022}
\author{Lee Farrugia}

\begin{document}
    
\maketitle
\thispagestyle{titlepagestyle}
\pagestyle{mystyle}

\section{Abstract}
yes

\begin{multicols*}{2}

\section{Introduction \& Theoretical Background}
The use of open-ended coaxial probes to measure the complex permittivity is used as it is a non invasive and non destructive to the materials being checked. However, those are not the only advantages this method has, but also due to the way it is utilised it allow for bandwidth measurement and easy sampling \parencite{liao2011accurate}. This method can be used to measure the microwave complex permittivity of dielectric materials including lossy dielectrics. As the relationship between the reflection and transmission is quite complex, the equivalent circuit method is used to simplify the relationship \parencite{stuchly1982equivalent}. This is quite useful when dealing with lossy dielectrics and biological tissues, as the biological tissues are not only made up from one type of material but, multiple types of tissues in one sample. Another example of the use of this method is explored in \cite{zajivcek2006evaluation}, where they discuss the relationship of the the complex permittivity and the transmission of healthy tissues and tumour tissues. They further discuss how this method could be used to image people with non-ionising radiation.

\section{Materials \& Methods}
\subsection{Language and Packages}
Python 3.10.8, Numpy, Pandas, Matplotlib.pyplot \,.
\subsection{Methodology}
yes

\section{Results \& Discussion}
yes

\end{multicols*}

\section{References}
\printbibliography[heading = none]


\end{document}