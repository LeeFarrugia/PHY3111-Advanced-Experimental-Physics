\documentclass[12pt, a4paper]{article}
\usepackage{import}
\usepackage{preamble}

\title{Pendulum Project}
\date{\(1^\mathrm{{st}}\) December 2022}
\author{Lee Farrugia}

\begin{document}
    
\maketitle
\thispagestyle{titlepagestyle}
\pagestyle{mystyle}

\section{Abstract}


\section{Introduction \& Theoretical Background}


\section{Materials \& Methods}
\subsection{Language and Packages}
Python 3.10.8, Numpy, Sympy, Scipy, Matplotlib.pyplot, Scipy.integrate, Scipy.fft, Matplotlib.patches

\subsection{Methodology}
\textbf{Single Compound Pendulum}
\begin{enumerate}
    \item Use the Sympy package to define the relations between the Cartesian coordinates, \textit{x} and
    \textit{y}, and the generalised coordinate \textit{\(\theta\)}. Determine the velocities in both the \textit{x} and \textit{y} directions. Furthermore, determine the rectilinear and rotational energies of the pendulum. By working out the total kinetic energy and potential energies of the system find the Lagrangian of the system. Assume that the zero-point for the potential energy is at the minimum heigh of the pendulum. Find the equation of motion using the Euler-Lagrange equation. 
    \item Using the following initial conditions: \(m = \qty{1}{\kilogram}, l = \qty{1}{\meter}, i = \qty{0.025}{\kilogram\metre\squared}\), calculate \(\theta\) and \(\dot{\theta}\) over time, taking samples of \qty{100}{\hertz} for \qty{10}{\second}, taking \(\theta = \frac{\pi}{2}\)\unit{\radian} and \(\dot{\theta}\)=\qty{0}{\radian\per\second}. Take a Fourier transformation of the oscillating system and create an animation of the pendulum.
\end{enumerate}
\textbf{Double Compound Pendulum}
\begin{enumerate}
    \item Using the same the Sympy package again define the relationship between the generalised coordinates, \(\theta_1\) and \(\theta_2\). Using the same steps as before determine the equations of motion for the double compound pendulum using the Euler-Lagrange equation. 
    \item  Using the following initial conditions: \(m_1 = \qty{2}{\kilogram}, m_2 = \qty{1}{\kilogram}, l_1 = \qty{1}{\meter}, l_2 = \qty{1.5}{\meter}, i_1 = \qty{0.025}{\kilogram\meter\squared}, i_2 = \qty{0.075}{\kilogram\meter\squared}\), calculate \(\theta\) and \(\dot{\theta}\), taking the same samples for the same amount of time as before and taking \(\theta_1 = \qty{0}{\radian}, \dot{\theta_1} = \qty{0}{\radian\per\second}, \theta_2 = \frac{\pi}{2}\unit{\radian}, \dot(\theta_2) = \qty{0}{\radian\per\second}\). Take a Fourier transformation and create another animation of the pendulum system. 
\end{enumerate}

\section{Results \& Discussion}
The simple compound pendulum equation of motion that was derived is as follows:

\begin{align}
    \ddot{\theta} = \frac{-mgr\sin \theta}{i+mr^2}
\end{align}

where m is the mass of the pendulum, theta is the angle, i is the inertia, r is the radius of the oscillations and \(\ddot{\theta}\) is the acceleration in theta. By taking the system properties and initial conditions as mentioned in the method, the graph~\ref{fig: pendulum 1.1} was obtained. In this graph one can see that both the change in \(\dot{\theta}\) and \(\ddot{\theta}\) is harmonic with time. This corresponds to the positional change that ne observes when observing a pendulum.

\begin{figure}[H]
    \centering
    \includegraphics[width = 0.7\textwidth]{plots/Plot 1.1.png}\caption{Plot of the outlined system properties and initial conditions}\label{fig: pendulum 1.1}
\end{figure}

However, if the system properties were kept constant and the initial conditions are changed, different change in \(\theta\) can be observed. For example if the initial conditions are changed to: \(\theta = \frac{\pi}{4}, \dot{\theta} = 0\), the following graph is obtained:

\begin{figure}[H]
    \centering
    \includegraphics[width = 0.7\textwidth]{plots/Plot 1.5.png}\caption{\(\theta = \frac{\pi}{4}, \dot{\theta} = 0\)}\label{fig: pendulum 1.2}
\end{figure}

In graph~\ref{fig: pendulum 1.2}, one can see that the amplitude does not reach the previous extent as can be seen in graph~\ref{fig: pendulum 1.1} as the initial angle is halved. After which the Fourier transformation was applied on the system in order to observe the change in \(\theta\) with frequency. If the outlined initial are used the following graph is obtained:

\begin{figure}[H]
    \centering
    \includegraphics[width = 0.7\textwidth]{plots/Plot 2.1.png}\caption{Fourier transformation of the initial conditions}\label{fig: foruier 1.1}
\end{figure}

Here, one can notice the initial high peak when the frequency is at \qty{10}{\hertz}, with additional smaller peaks at higher frequencies. This corresponds to the sampling frequency of \qty{100}{\hertz} in \qty{10}{\second}. However, if the conditions are changed to: \(\theta = \frac{\pi}{4}, \dot{\theta} = 0\), the following graph is obtained:

\begin{figure}[H]
    \centering
    \includegraphics[width = 0.7\textwidth]{plots/Plot 2.5.png}\caption{Fourier transformation of \(\theta = \frac{\pi}{4}, \dot{\theta} = 0\)}\label{fig: fourier 1.2}
\end{figure}

One should observe a smaller high peak at ten seconds and less smaller peaks as the frequency increases. This, sheds light on the fact that with different initial conditions the amplitude will change accordingly and different peaks will be observed. All of these results correspond with the observation of a simple compound pendulum as mentioned in \cite{pedersen1977compound}. 

Next the double compound pendulum was simulated with the given system properties and initial conditions. From this simulation the following graph was obtained:

\begin{figure}[H]
    \centering
    \includegraphics[width = 0.7\textwidth]{plots/Plot 3.1.png}\caption{A graph of the given system properties and initial conditions for the double compound pendulum}\label{fig: pendulum 2.1}
\end{figure}

Here, one should note that the usual harmonic motion of the pendulum is lost in both parts opf the pendulum. Additionally not only is the harmonic nature lost but it is extremely lost in the second part of the pendulum. After this a Fourier transformation is applied on the initial system properties and conditions, in order to obtain the following graph:

\begin{figure}[H]
    \centering
    \includegraphics[width = 0.7\textwidth]{plots/Plot 3.2.png}\caption{Fourier transformation the initial conditions for double compound pendulum}\label{fig: fourier 2.2}
\end{figure}

In this Fourier transformation one should note that there are two peaks at the \qty{100}{\hertz}, one for each pendulum part. Additionally one should not the increased amount of smaller peaks at higher frequency, which corresponds to the motion of the double compound pendulum. Furthermore, one should additionally not that at larger frequencies the second part of pendulum denoted as pendulum 2 in graph~\ref{fig: fourier 3.2}, have more peaks than the first part of the pendulum. 

As stated in \cite{chaos}, the double compound pendulum relies heavily on the initial conditions, as a small change in these will effect the outcome. However, it should be noted that nonetheless double compound pendulum suffers from what is referred to chaos. As mentioned before a change in the initial conditions, will result in a change in the chaos of the system. This chaos can be seen in graph~\ref{fig: pendulum 1.1} where the second part of the pendulum that is denoted as pendulum 2, instantly loses any form of harmonic motion, which in turn will affect the motion of the first of the pendulum. \cite{chaos} also state that if two double pendulums are put in motion together with the same angle they will continue move in tandem but at ta point will they will no longer be in tandem. This separation in movement can be seen earlier if the initial angle is even larger. They, further stress that this is not due to any experimental set up mistakes but rather due to the chaotic nature of the pendulums.

\section{Conclusion}


\section{References}
\printbibliography[heading = none]

\section{Appendix}
\begin{minted}{python}
import numpy as np
import scipy as sc
import matplotlib.pyplot as plt
from math import sin, cos
from sympy import *
from scipy.integrate import odeint
from scipy.fft import fft, fftfreq
from matplotlib.patches import Circle

x,y,r,t,m,v,i,g = symbols('x,y,r,t,m,v,i,g')
theta = Function('theta')(t)
x = r * sin(theta)
y = (-1*r) * cos(theta)
theta_dot = diff(theta, t)

vx = diff(x,t)
vy = diff(y,t)

vt = vx**2 + vy**2
T_rec = 0.5* m* v.subs(v, vt)
T_rot = 0.5* i* diff(theta,t)**2

T = T_rec + T_rot
U = m*g*r*(1 - cos(theta))
L = T - U

Q = diff(diff(L, theta_dot),t) - diff(L, theta)

def ode_func(theta,t,m,g,r,i):
    #theta
    theta1=theta[0]
    #theta_dot
    dtheta1_dt=theta[1]
    #second ode
    dtheta2_dt = (-m*g*r*(np.sin(theta1)))/(i+(m*(r**2)))
    dtheta_dt = [dtheta1_dt, dtheta2_dt]
    return dtheta_dt


g = 9.81 # acceleration due to the gravity
i = 0.025
r = 0.5 # radius of pendulum
m = 1 # mass 
l = 1 # length of pendulum  

# initial conditions
theta_0 = [(np.pi/2),0, (np.pi/2), (np.pi/2), (np.pi/4), 0, (np.pi/4)]
# time plot
t = np.linspace(0, 10, 1000)

# solving the ode
for a in range(len(theta_0)-1):
    theta = odeint(ode_func,theta_0[a:a+2],t,args=(m,g,r,i))

    plt.figure(figsize=(7.5, 10.5))
    plt.rcParams['font.family'] = 'STIXGeneral'
    plt.rcParams['mathtext.fontset'] = 'stix'
    plt.rcParams['font.size'] = 12
    plt.rcParams['font.weight'] = 'normal'
    plt.minorticks_on()
    plt.grid(visible=True, which='major', linestyle='-')
    plt.grid(visible=True, which='minor', linestyle='--')
    plt.xlim(0,10)

    plt.plot(t,theta[:,0],'--', color='k', label=r'$\dot{\mathrm{\theta}}$')
    plt.plot(t,theta[:,1], color='k', label=r'$\mathrm{\ddot{\theta}}$')
    plt.xlabel('t/s')
    plt.ylabel(r'$\Delta \theta$/rads')
    plt.title(r'A graph of the change in $\mathrm{\theta}$ in time')

    plt.legend()
    plt.tight_layout()
    plt.savefig(f'plots/Plot 1.{a+1}.png', dpi=800)
    plt.close()

    # Fourier transformation
    # Sampling space = Frequency * total time
    N = 100 * 10
    # theta
    x = theta[:,0]
    # theta_dot
    y = theta[:,1]
    # Periodic Time
    T = 1/N
    yf = fft(y)
    xf = fftfreq(N, T)[:N//2]

    # Plotting conditions
    plt.figure(figsize=(7.5, 10.5))
    plt.rcParams['font.family'] = 'STIXGeneral'
    plt.rcParams['mathtext.fontset'] = 'stix'
    plt.rcParams['font.size'] = 12
    plt.rcParams['font.weight'] = 'normal'
    plt.minorticks_on()
    plt.grid(visible=True, which='major', linestyle='-')
    plt.grid(visible=True, which='minor', linestyle='--')
    # Plotting the data
    plt.plot(xf, 2/N * np.abs(yf[0:(N//2)]), 'k')
    plt.xlabel('Frequency')
    plt.ylabel(r'$\Delta \theta$/rads')
    plt.title('Fourier transformation of the Single Pendulum')
    plt.tight_layout()
    # Saving the figure
    plt.savefig(f'plots/Plot 2.{a+1}.png', dpi=800)
    plt.close()

plt.figure()

theta = odeint(ode_func,theta_0[0:2],t,args=(m,g,r,i))
theta_a = theta[:,0]
x = r * np.sin(theta_a)
y = (-1*r) * np.cos(theta_a)

tmax, dt = 10, 0.01
t1 = np.arange(0, tmax+dt, dt)

r = 0.05
trail_secs = 1
max_trail = int(trail_secs / dt)

def make_plot1(i):
    ax.plot([0, x[i]], [0, y[i]], lw=2, c='k')
    c0 = Circle((0, 0), r/2, fc='k', zorder=10)
    c1 = Circle((x[i], y[i]), r, fc='b', ec='b', zorder=10)
    ax.add_patch(c0)
    ax.add_patch(c1)

    ns = 20
    s = max_trail // ns

    for j in range(ns):
        imin = i - (ns-j)*s
        if imin < 0:
            continue
        imax = imin + s + 1
        alpha = (j/ns)**2
        ax.plot(x[imin:imax], y[imin:imax], c='r', solid_capstyle='butt', lw=2, alpha=alpha)

    ax.set_xlim(-l-r, l+r)
    ax.set_ylim(-l-r, l+r)
    ax.set_aspect('equal', adjustable='box')
    plt.axis('off')
    plt.savefig('frames1/_img{:04d}.png'.format(i//di), dpi=100)
    plt.cla()


fps = 10
di = int(1/fps/dt)
fig = plt.figure(figsize=(8.3333, 6.25), dpi=72)
ax = fig.add_subplot(111)

for i in range(0, t1.size-1, di):
    make_plot1(i)

x1,y1,x2,y2,l1,l2,v1,v2,t,m1,m2,i1,i2,g = symbols('x1,y1,x2,y2,l1,l2,v1,v2,t,m1,m2,i1,i2,g')
theta1 = Function('theta1')(t)
theta2 = Function('theta2')(t)

theta_dot1 = diff(theta1, t)
theta_dot2 = diff(theta2, t)

x1 = (l1/2)*sin(theta1)
x2 = l1*sin(theta1) + (l2/2)*sin(theta2)
y1 = (-1*(l1/2))*cos(theta1)
y2 = (-1*(l1*cos(theta1)))-(l2/2)*cos(theta2)

vx1 = diff(x1, t)
vx2 = diff(x2, t)
vy1 = diff(y1, t)
vy2 = diff(y2, t)

vt1 = vx1**2 + vy1**2
vt2 = vx2**2 + vy2**2

Trec1 = 0.5 * m1 * v1.subs(v1, vt1)
Trec2 = 0.5 * m2 * v1.subs(v2, vt2)

Trot1 = 0.5* i1* diff(theta1,t)**2
Trot2 = 0.5* i2* diff(theta2,t)**2

T = Trec1 + Trot1 +Trec2 + Trot2
U1 = ((m1*g*l1)/2)*1-cos(theta1)
U2 = ((m2*g)*((l1*(1-cos(theta1))))+((l2/2)*(1-cos(theta2))))
U = U1 + U2
L = T - U

Q1 = diff(diff(L,theta_dot1),t)-diff(L,theta1)
Q2 = diff(diff(L,theta_dot2),t)-diff(L,theta2)

l1, l2 = 1, 1.5
m1, m2 = 2, 1
g = 9.81

def ode_func_2(y, t, l1, l2, m1, m2):
    theta1, z1, theta2, z2 = y

    c, s = np.cos(theta1-theta2), np.sin(theta1-theta2)

    theta1dot = z1
    z1dot = (m2*g*np.sin(theta2)*c - m2*s*(l1*z1**2*c + l2*z2**2) - (m1+m2)*g*np.sin(theta1)) / l1 / (m1 + m2*s**2)
    theta2dot = z2
    z2dot = ((m1+m2)*(l1*z1**2*s - g*np.sin(theta2) + g*np.sin(theta1)*c) + m2*l2*z2**2*s*c) / l2 / (m1 + m2*s**2)
    return theta1dot, z1dot, theta2dot, z2dot

def total_E(y):
    th1, th1d, th2, th2d = y.T
    V = -(m1+m2)*l1*g*np.cos(th1) - m2*l2*g*np.cos(th2)
    T = 0.5*m1*(l1*th1d)**2 + 0.5*m2*((l1*th1d)**2 + (l2*th2d)**2 + 2*l1*l2*th1d*th2d*np.cos(th1-th2))
    return T + V

tmax, dt = 10, 0.01
t = np.arange(0, tmax+dt, dt)
y0 = np.array([0, 0, np.pi/2, 0])

y = odeint(ode_func_2, y0, t, args=(l1, l2, m1, m2))

theta1, theta2 = y[:,0], y[:,2]

plt.close('all')

plt.figure(figsize=(7.5, 10.5))
plt.rcParams['font.family'] = 'STIXGeneral'
plt.rcParams['mathtext.fontset'] = 'stix'
plt.rcParams['font.size'] = 12
plt.rcParams['font.weight'] = 'normal'
plt.minorticks_on()
plt.grid(visible=True, which='major', linestyle='-')
plt.grid(visible=True, which='minor', linestyle='--')

plt.plot(t, theta1, color='k', label='Pendulum 1')
plt.plot(t, theta2,'--' , color='k', label='Pendulum 2')
plt.xlabel('t/s')
plt.ylabel(r'$\Delta \theta$/rads')
plt.title(r'A graph of the change in $\mathrm{\theta}$ in time')
plt.legend()
plt.savefig('plots/Plot 3.1.png', dpi=800)

plt.close('all')

x1 = l1 * np.sin(theta1)
y1 = -l1 * np.cos(theta1)
x2 = x1 + l2 * np.sin(theta2)
y2 = y1 - l2 * np.cos(theta2)

r = 0.05
trail_secs = 1
max_trail = int(trail_secs / dt)

def make_plot2(i):
    ax.plot([0, x1[i], x2[i]], [0, y1[i], y2[i]], lw=2, c='k')
    c0 = Circle((0, 0), r/2, fc='k', zorder=10)
    c1 = Circle((x1[i], y1[i]), r, fc='b', ec='b', zorder=10)
    c2 = Circle((x2[i], y2[i]), r, fc='r', ec='r', zorder=10)
    ax.add_patch(c0)
    ax.add_patch(c1)
    ax.add_patch(c2)

    ns = 20
    s = max_trail // ns

    for j in range(ns):
        imin = i - (ns-j)*s
        if imin < 0:
            continue
        imax = imin + s + 1
        alpha = (j/ns)**2
        ax.plot(x2[imin:imax], y2[imin:imax], c='r', solid_capstyle='butt', lw=2, alpha=alpha)

    ax.set_xlim(-l1-l2-r, l1+l2+r)
    ax.set_ylim(-l1-l2-r, l1+l2+r)
    ax.set_aspect('equal', adjustable='box')
    plt.axis('off')
    plt.savefig('frames2/_img{:04d}.png'.format(i//di), dpi=100)
    plt.cla()


fps = 10
di = int(1/fps/dt)
fig = plt.figure(figsize=(8.3333, 6.25), dpi=72)
ax = fig.add_subplot(111)

for i in range(0, t.size, di):
    make_plot2(i)

def fourier(theta1,theta2):
    # Fourier transformation
    # Sampling space = Frequency * total time
    N = 100 * 10
    # theta
    x1 = theta1
    # theta_dot
    y1 = theta1
    # Periodic Time
    T = 1/N
    yf1 = fft(y1)
    xf1 = fftfreq(N, T)[:N//2]

    # theta
    x2 = theta2
    # theta_dot
    y2 = theta2
    # Periodic Time
    T = 1/N
    yf2 = fft(y2)
    xf2 = fftfreq(N, T)[:N//2]

    plt.figure(figsize=(7.5, 10.5))
    plt.rcParams['font.family'] = 'STIXGeneral'
    plt.rcParams['mathtext.fontset'] = 'stix'
    plt.rcParams['font.size'] = 12
    plt.rcParams['font.weight'] = 'normal'
    plt.minorticks_on()
    plt.grid(visible=True, which='major', linestyle='-')
    plt.grid(visible=True, which='minor', linestyle='--')

    plt.plot(xf1, 2/N * np.abs(yf1[0:(N//2)]), 'k', label='Pendulum 1')
    plt.plot(xf2, 2/N * np.abs(yf2[0:(N//2)]), label='Pendulum 2')
    plt.xlabel('Frequency')
    plt.ylabel(r'$\Delta \theta$/rads')
    plt.title('Fourier transformation of the Double Pendulum')
    plt.legend()
    # Saving the figure
    plt.savefig('plots/Plot 3.2.png', dpi=800)
    plt.close()

fourier(theta1, theta2)

\end{minted}

\end{document}